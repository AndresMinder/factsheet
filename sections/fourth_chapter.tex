\NewsItem{Vor- und Nachteile des Taupunktspiegelverfahrens}

\begin{tabular}{l|l}
\textbf{Vorteile} & \textbf{Nachteile} \\ 
\hline 
Im ganzen messtechnisch interessierenden Bereich einsetzbar & Muss vom Luftstrom durchflutet werden \\ 
\hline 
Zählen zu den genauesten Hygrometern & Messgenauigkeit hängt von der Genauigkeit der \\&Spiegeltemperaturmessung und von der Güte der \\& Regelung ab  \\ 
\hline 
Als Referenzmittel geeignet & Anfällig gegen mechanische Verschmutzung \\ 
\end{tabular}\\
\begin{multicols}{3} 
Wie in der Tabelle zu sehen ist, hängt die Genauigkeit dieses Verfahrens von der Messgenauigkeit der Spiegeltemperaturmessung und von der Güte der Regelung der Spiegeltemperatur ab. Aus diesem Grund kann keine quantifizierte Zahl für die Genauigkeit genannt werden. Prinzipiell kann mit diesem Verfahren jedoch eine so hohe Genauigkeit erreicht werden, dass Referenzmittel dieses Verfahren nutzen.\\
Wie bereits erwähnt misst ein optisches Element die Reflexionsverhältnisse des Spiegels. Wenn nun die Feuchtigkeit der Luft oder eines Gases gemessen werden soll, muss darauf geachtet werden, dass keine anderen kondensierbare Komponenten in den Luftstrom gelangen, da diese die Detektion und somit die Messung beeinflussen.
\end{multicols}